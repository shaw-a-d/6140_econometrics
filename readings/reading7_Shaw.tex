% Options for packages loaded elsewhere
\PassOptionsToPackage{unicode}{hyperref}
\PassOptionsToPackage{hyphens}{url}
\documentclass[
]{article}
\usepackage{xcolor}
\usepackage[margin=1in]{geometry}
\usepackage{amsmath,amssymb}
\setcounter{secnumdepth}{-\maxdimen} % remove section numbering
\usepackage{iftex}
\ifPDFTeX
  \usepackage[T1]{fontenc}
  \usepackage[utf8]{inputenc}
  \usepackage{textcomp} % provide euro and other symbols
\else % if luatex or xetex
  \usepackage{unicode-math} % this also loads fontspec
  \defaultfontfeatures{Scale=MatchLowercase}
  \defaultfontfeatures[\rmfamily]{Ligatures=TeX,Scale=1}
\fi
\usepackage{lmodern}
\ifPDFTeX\else
  % xetex/luatex font selection
\fi
% Use upquote if available, for straight quotes in verbatim environments
\IfFileExists{upquote.sty}{\usepackage{upquote}}{}
\IfFileExists{microtype.sty}{% use microtype if available
  \usepackage[]{microtype}
  \UseMicrotypeSet[protrusion]{basicmath} % disable protrusion for tt fonts
}{}
\makeatletter
\@ifundefined{KOMAClassName}{% if non-KOMA class
  \IfFileExists{parskip.sty}{%
    \usepackage{parskip}
  }{% else
    \setlength{\parindent}{0pt}
    \setlength{\parskip}{6pt plus 2pt minus 1pt}}
}{% if KOMA class
  \KOMAoptions{parskip=half}}
\makeatother
\usepackage{graphicx}
\makeatletter
\newsavebox\pandoc@box
\newcommand*\pandocbounded[1]{% scales image to fit in text height/width
  \sbox\pandoc@box{#1}%
  \Gscale@div\@tempa{\textheight}{\dimexpr\ht\pandoc@box+\dp\pandoc@box\relax}%
  \Gscale@div\@tempb{\linewidth}{\wd\pandoc@box}%
  \ifdim\@tempb\p@<\@tempa\p@\let\@tempa\@tempb\fi% select the smaller of both
  \ifdim\@tempa\p@<\p@\scalebox{\@tempa}{\usebox\pandoc@box}%
  \else\usebox{\pandoc@box}%
  \fi%
}
% Set default figure placement to htbp
\def\fps@figure{htbp}
\makeatother
\setlength{\emergencystretch}{3em} % prevent overfull lines
\providecommand{\tightlist}{%
  \setlength{\itemsep}{0pt}\setlength{\parskip}{0pt}}
\usepackage{bookmark}
\IfFileExists{xurl.sty}{\usepackage{xurl}}{} % add URL line breaks if available
\urlstyle{same}
\hypersetup{
  pdftitle={Reading Assignment 7},
  pdfauthor={Andrew Shaw},
  hidelinks,
  pdfcreator={LaTeX via pandoc}}

\title{Reading Assignment 7}
\author{Andrew Shaw}
\date{2025-10-12}

\begin{document}
\maketitle

\section{Angrist \& Pischke Chapter 2}\label{angrist-pischke-chapter-2}

\subsection{1. What is the ingenious matching exercise in table 2.1? Why
can we do this and forget about ALL possible omitted
variables?}\label{what-is-the-ingenious-matching-exercise-in-table-2.1-why-can-we-do-this-and-forget-about-all-possible-omitted-variables}

By taking all observations in where there is uniformity across
measurable characteristics so that each subgroup can be compared while
looking only at the variable of interest's effect on the dependent
outcome thereby reduce any influence from other characteristics on the
outcome. Then subgroups can then be compared using weights depending on
the size of those subgroups. This is a ceteris paribus situation so that
comparisons are more like-with-like.

\subsection{2. Why would it not have been prudent to compare students in
Group A to those in Group
D?}\label{why-would-it-not-have-been-prudent-to-compare-students-in-group-a-to-those-in-group-d}

The question being posed is whether a private school education has an
effect on outcome later in life. Tho students in group D were all of the
control group having only been accepted to public schools so there is no
private school effect to try and isolate within group let alone across
groups to group A.

\subsection{3. Why does all of the weight of the regression to compare
public to private rest on the effects in Group A and B and none of the
weight come from comparisons within group C and group
D?}\label{why-does-all-of-the-weight-of-the-regression-to-compare-public-to-private-rest-on-the-effects-in-group-a-and-b-and-none-of-the-weight-come-from-comparisons-within-group-c-and-group-d}

Again for similar reasons above, the two groups only offer information
about either the control or the treatment group and thus there is no
comparison to be made regarding benefits of private education.

\subsection{4. Why are the weights different in regression than you'd
think they might be if you look at simply the number of observations in
the
group?+}\label{why-are-the-weights-different-in-regression-than-youd-think-they-might-be-if-you-look-at-simply-the-number-of-observations-in-the-group}

The addition of the group variable in the regression means the weight no
looks within group at the variation between private and public schools
not across group so the simple average is no longer sufficient for
weighting.

\subsection{5. What does transform the left-hand side of an equation by
taking the natural log do to the beta�s on the right-hand
side?}\label{what-does-transform-the-left-hand-side-of-an-equation-by-taking-the-natural-log-do-to-the-betas-on-the-right-hand-side}

This changes the interpretation of the coefficient to represent a
percentage change in the dependent variable for a one unit change in the
regressor.

\subsection{6. How can I use regression to get the price elasticity of
demand explicitly from regression
output?}\label{how-can-i-use-regression-to-get-the-price-elasticity-of-demand-explicitly-from-regression-output}

Tranform both the dependent variable and the regressor into natural log
format and then interpretation of the beta coefficient now becomes the
elasticity.

\subsection{7. What can we learn from table 2.2. Explain the results in
detail? Does private school matter as much as many assume it
does?}\label{what-can-we-learn-from-table-2.2.-explain-the-results-in-detail-does-private-school-matter-as-much-as-many-assume-it-does}

As more variables are included in the model and effects are isolated,
the private school premium continues to decline while remaining
stastistically significant.

\subsection{8. Explain why we might be looking at a limited set here?
What are the public schools included and how might they compare to other
public
schools?}\label{explain-why-we-might-be-looking-at-a-limited-set-here-what-are-the-public-schools-included-and-how-might-they-compare-to-other-public-schools}

Because we're only looking at schools featured in the Barron's list of
selectivity.

\section{9. Why might table 2.4 shed a little more light on the
problem?}\label{why-might-table-2.4-shed-a-little-more-light-on-the-problem}

Table shifts to looking at the campus-wide avg. SAT score of students
which is continuous as opposed to a binary variable of private v.
public. The binary variable captures a lot of information that isn't
necessarily specific to private schools.

\subsection{10. We can't have all omitted variables? Why does
sensitivity analysis give us greater confidence that we haven't omitted
anything
important?}\label{we-cant-have-all-omitted-variables-why-does-sensitivity-analysis-give-us-greater-confidence-that-we-havent-omitted-anything-important}

By comparing two models, one with a robust amount of controls and
simpler model, the difference in the beta of the regressor of interest
doesn't reveal a drastic difference there is unlikely a missing variable
that could cause large fluctuations.

\subsection{11. How can we quantify the omitted variable bias
differences when we include key controls compared to when we omit key
controls?}\label{how-can-we-quantify-the-omitted-variable-bias-differences-when-we-include-key-controls-compared-to-when-we-omit-key-controls}

Take the \(\hat{\beta_S} - \hat{\beta_L} = \deltaOVB\). There are other
robustness checks which can be run as well, this being the most basic.

\subsection{12. Describe from table 2.5 what we can learn about omitted
variable bias in this
context?}\label{describe-from-table-2.5-what-we-can-learn-about-omitted-variable-bias-in-this-context}

The table shows that earnings gap resulting from attending private
school was actually bias coming from family background (parental income)
and SAT scores of individuals.

\subsection{13. Why might we still have omitted variable bias despite
our
calculations?}\label{why-might-we-still-have-omitted-variable-bias-despite-our-calculations}

The calculations only describe how sensitive the coefficients are to
hypothetical missing variables so there could be additional OVB not
considered in the model.

\subsection{14. What is regression to the mean? How was it found as a
phenomenon?}\label{what-is-regression-to-the-mean-how-was-it-found-as-a-phenomenon}

It's a stastistical property of correlated pairs of variables discovered
when Galton studied heights of fathers and sons and compared to average
height of population. He noticed that shorter than average parents tend
to have slightly taller children and taller than average parents have
slightly shorter children.

\end{document}
